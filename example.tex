\begin{table}
 \caption{Beautiful caption}
 \label{tab:important_label}
 \centering
\sisetup{detect-all}
 \begin{tabular}{SS}
 \toprule 
    {title column 1}& {title column 2} \\
     \midrule
           2.00 & $\num{1.096\pm0.011}$ \\
           3.00 & $\num{2.0000\pm0.0024}$ \\
           5.00 & $\num{3.00\pm0.20}$ \\
           6.00 & $\text{\textbf{---}}$ \\
           7.00 & $\text{\textbf{---}}$ \\
 \bottomrule
 \end{tabular}
\end{table}